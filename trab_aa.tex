\documentclass[10pt,a4paper]{article}
\usepackage[utf8]{inputenc}
\usepackage[portuguese]{babel}
\usepackage[T1]{fontenc}
\usepackage{amsmath}
\usepackage{amsfonts}
\usepackage{amssymb}
\usepackage{geometry}
\usepackage{tabto}


\begin{document}


\begin{flushleft}

	\large{Análise de Algoritmos - Trabalho 1}\\
	Clara de Mattos Szwarcman - 1310351\\
	Lucas Ribeiro Borges - \\
	Guilherme Simas Abinader -\\ 

\end{flushleft}


	\section*{1 - Controle de Qualidade na Produção de Frascos de Vidro}
	
		\vspace{1cm}
	
		\subsection*{1)}
		
		\tab A altura será dividida em raiz de n intervalos, aonde cada intervalo possui raiz de n degraus. O primeiro frasco será jogado de raiz de n em raiz de n degraus. Quando o frasco quebrar, jogaremos o segundo frasco a partir do início desse intervalo de degrau em degrau até que ele quebre.\\
		
		\textbf{Pseudo Código: }\\
		
		$Degrau\_2\_frascos(x,n)$\\

		\hspace{1cm} $raiz\_n = sqrt(n);$\\


		\hspace{1cm} $for$  $i = 0;$ $i < n;$ $i+=raiz\_n$\\

		\hspace{2cm} $if$ $i >= x$\\

		\hspace{3cm} $for$ $j = i-raiz_n;$ $j < i;$ $j++$\\

		\hspace{4cm} $if$ $j == x$\\

		\hspace{5cm} $return$ $j;$\\


\vspace{1cm}

	Quando o primeiro frasco quebrar teremos um intervalo de tamanho $ \sqrt{n} $ que com certeza contém a altura em que o frasco quebra, visto que se o frasco não quebrou no degrau anterior ao início do intervalo, ele também não quebra em nenhum dos degraus abaixo dele. Assim, ao percorrermos o intervalo de um em um, encontraremos a altura x.\\
	
	No pior dos casos, o frasco quebra no último degrau, portanto, o primeiro frasco será jogado de todos os intervalos $(\sqrt{n}$ vezes.$)$ Como ele quebrou no último degrau, será conferido se ele quebra em algum degrau pertencente ao último intervalo. Dessa maneira, o segundo frasco será jogado em cada degrau do intervalo $(\sqrt{n}$ vezes$)$, até finalmente quebrar no último degrau. Sendo assim, foi jogado $2*\sqrt{n}$.\\
	
	\begin{center}
		$O(2*\sqrt{n}) = O(\sqrt{n})$	
	
	\end{center}
	
	
	\subsection*{2)}
	
	\vspace{0.5cm}
	
	\tab Tendo 3 frascos:\\

\hspace{1cm} O primeiro frasco será jogado em intervalos de $n^{\frac{2}{3}}$ até quebrar. O segundo frasco será jogado em intervalos de $n^\frac{1}{3}$, no intervalo de $n^{\frac{2}{3}}$ encontrado. O terceiro frasco será jogado de degrau em degrau no intervalo de $n^{\frac{1}{3}}$ encontrado.\\


Tendo 4 frascos:\\

\hspace{1cm} O primeiro frasco será jogado em intervalos de $n^{\frac{3}{4}}$ até quebrar. O segundo frasco será jogado em intervalos de $n^{\frac{2}{4}}$, no intervalo de $n^{\frac{3}{4}}$ encontrado. O terceiro frasco será jogado em intervalos de $n^{\frac{1}{4}}$, no intervalo de $n^{\frac{2}{4}}$ encontrado. O quarto frasco será jogado de degrau em degrau no intervalo de $n^{\frac{1}{4}}$ encontrado.\\

Tendo k frascos:\\

 \hspace{1cm} A altura $((\sqrt[k]{n})^{k}$ degraus$)$é dividida em $\sqrt[k]{n}$ intervalos de tamanho $(\sqrt[k]{n})^{k-1}$. Jogamos o primeiro frasco em intervalos de $(\sqrt[k]{n})^{k-1}$ degraus. Quando o frasco quebrar, o último intervalo $((\sqrt[k]{n})^{k-1}$ degraus$)$ será dividido em $\sqrt[k]{n}$ intervalos de tamanho $(\sqrt[k]{n})^{k-2}$ degraus e o segundo frasco será jogado em intervalos de $(\sqrt[k]{n})^{k-2}$ degraus. Isso ocorrerá sucessivamente para todos os k frascos. No frasco k teremos um intervalo de tamanho $(\sqrt[k]{n})^{k-(k-1)}$, que é igual a $\sqrt[k]{n}$. O frasco será jogado em intervalos de $(\sqrt[k]{n})^{k-k}$, ou seja, de degrau em degrau, até quebrar.\\ 

 
\textbf{Pseudo Código: }\\


	$Degrau\_k\_frascos(x,n,k)$\\
	
	\hspace{1cm}$raiz\_kesima = raiz(n,k)$\\
	
	\hspace{1cm}$inicio = 0;$\\
	
	\hspace{1cm}$fim = n;$\\
	
	\hspace{1cm}$incremento = pow(raiz\_kesima,k-1)$\\
	
	\hspace{1cm}$for$ $i=0;$ $i < k;$ $i++$\\
	
	\hspace{2cm}$for$ $j = inicio;$ $j < fim;$ $j+= incremento$\\
	
	\hspace{3cm}$if$ $j >= x$\\
	
	\hspace{4cm}$if$ $incremento == 1$\\
	
	\hspace{5cm}$return$ $j;$\\
	
	\hspace{4cm}$inicio = j - incremento ;$\\
	
	\hspace{4cm}$fim = j;$\\
	
	\hspace{4cm}$incremento = incremento/raiz\_kesima;$\\
	
	\hspace{4cm}$break;$\\
	
	
	Segue a premissa do primeiro, aonde sempre se tem certeza do intervalo em que ocorre a quebra, porém com mais frascos para serem utilizados. Portanto, podemos inicialmente dividir a altura em intervalos maiores com mais subdivisões, assim postergando a procura de um em um, que será feita em um intervalo menor.\\
	
	Para cada frasco estamos realizando no máximo $\sqrt[k]{n}$ testes, visto que para o frasco i temos um espaço de $(\sqrt[k]{n})^{k-(i-1)}$ degraus e o jogaremos em intervalos de $(\sqrt[k]{n})^{k-i}$ degraus. Como temos k frascos, isso será realizado k vezes. Assim, o número total de quedas será $k * \sqrt[k]{n}$.\\
	
	\begin{center}
	
		$O(k*\sqrt[k]{n})$\\
	\end{center}
	
	Se k, é um número fixo, a complexidade será O($\sqrt[k]{n}$), como provado anteriormente.\\
	
	
	\subsection*{3)}
	
	\vspace{0.5cm}
	
	\tab A menor complexidade assintótica possível é de $O(log n)$.\\

O algoritmo realiza uma busca binária ao longo da escada, jogando um frasco a cada comparação, se o frasco quebra, busca-se na metade inferior, do contrário busca-se na metade superior. Quando o intervalo é de 1 degrau, podemos garantir que encontramos a altura correta.\\


\textbf{Pseudo Código: }\\

$Degrau\_logn\_frascos(x,n)$\\
					

	\hspace{1cm}$busca\_binaria(x,n)$
	
	

	\subsection*{Resultados}

	\vspace{0.5cm}	
	
	\begin{center}
		\begin{tabular}{|c|c|c|c|}
			\hline
			Nome do Arquivo & Degraus & Frascos & cpu time $($ms$)$ \\
			\hline
			bignum$\_$32$\_$01 & $2^{32}$ & 1 & 7115744.463ms \\
			\hline
			bignum$\_$32$\_$01 & $2^{32}$ & 2 & 194.01299999999998ms \\
			\hline
			bignum$\_$32$\_$01 & $2^{32}$ & 4 & 1.7309999999999999ms \\
			\hline
			bignum$\_$32$\_$01 & $2^{32}$ & 8 & 0.40499999999999914ms \\
			\hline
			bignum$\_$32$\_$01 & $2^{32}$ & 16 & 0.4539999999999995ms \\
			\hline
			bignum$\_$32$\_$01 & $2^{32}$ & 32 & 0.7480000000000004ms \\
			\hline
			bignum$\_$32$\_$02 & $2^{32}$ & 1 & 5761534.251999999ms \\
			\hline
			bignum$\_$32$\_$02 & $2^{32}$ & 2 & 198.36ms \\
			\hline
			bignum$\_$32$\_$02 & $2^{32}$ & 4 & 1.7560000000000007ms \\
			\hline
			bignum$\_$32$\_$02 & $2^{32}$ & 8 & 0.3969999999999998ms \\
			\hline
			bignum$\_$32$\_$02 & $2^{32}$ & 16 & 0.4469999999999995ms \\
			\hline
			bignum$\_$32$\_$02 & $2^{32}$ & 32 & 0.7430000000000005ms \\
			\hline
			bignum$\_$64$\_$01 & $2^{64}$ & 2 & 13550835.350000001ms \\
			\hline
			bignum$\_$64$\_$01 & $2^{64}$ & 4 & 409.81199999999995ms \\
			\hline
			bignum$\_$64$\_$01 & $2^{64}$ & 8 & 3.5249999999999986ms \\
			\hline
			bignum$\_$64$\_$01 & $2^{64}$ & 16 & 0.7790000000000001ms \\
			\hline
			bignum$\_$64$\_$01 & $2^{64}$ & 32 & 0.8859999999999997ms \\
			\hline
			bignum$\_$64$\_$02 & $2^{64}$ & 4 & 429.72499999999997ms \\
			\hline
			bignum$\_$64$\_$02 & $2^{64}$ & 8 & 4.684999999999998ms \\
			\hline
			bignum$\_$64$\_$02 & $2^{64}$ & 16 & 0.9440000000000004ms \\
			\hline
			bignum$\_$64$\_$02 & $2^{64}$ & 32 & 1.0919999999999992ms \\
			\hline
			bignum$\_$128$\_$01 & $2^{128}$ & 8 & 898.033ms\\
			\hline
			bignum$\_$128$\_$01 & $2^{128}$ & 16 & 9.008999999999997ms\\
			\hline
			bignum$\_$128$\_$01 & $2^{128}$ & 32 & 2.252999999999998ms\\
			\hline
			bignum$\_$128$\_$02 & $2^{128}$ & 8 & 896.0749999999999ms\\
			\hline
			bignum$\_$128$\_$02 & $2^{128}$ & 16 & 9.394ms\\
			\hline
			bignum$\_$128$\_$02 & $2^{128}$ & 32 & 1.906999999999999ms\\
			\hline
			bignum$\_$192$\_$01 & $2^{192}$ & 16 & 125.40499999999999ms\\
			\hline
			bignum$\_$192$\_$01 & $2^{192}$ & 32 & 5.802000000000001ms\\
			\hline
			bignum$\_$192$\_$02 & $2^{192}$ & 16 & 125.98100000000001ms\\
			\hline
			bignum$\_$192$\_$02 & $2^{192}$ & 32 & 5.7680000000000025ms\\
			\hline
			bignum$\_$256$\_$01 & $2^{256}$ & 16 & 1973.921ms\\
			\hline
			bignum$\_$256$\_$01 & $2^{256}$ & 32 & 19.016ms\\
			\hline
			bignum$\_$256$\_$02 & $2^{256}$ & 16 & 1875.7930000000001ms\\
			\hline
			bignum$\_$256$\_$02 & $2^{256}$ & 32 & 19.178ms\\
			\hline
			
		\end{tabular}
	\end{center}
	
	
	
	\section*{2. Problema da Mochila Fracionária (pode-se colocar parte de um objeto na mochila)}
	
	\vspace{1cm}
	
		\subsection*{1.a)}
		
		\vspace{0.5cm}
	
	\tab Os objetos são ordenados por seu valor por peso com merge sort. São então adicionados a mochila de um em um começando pelo com maior valor por peso, até a capacidade ser atingida.\\
	
	\textbf{Pseudo Código: }\\
	
	$struct$ $objeto${\\
	
	\hspace{1cm}$int$ $valor;$\\
	
	\hspace{1cm}$int$ $peso;$\\
	
	\hspace{1cm}$float$ $densidade;$\\
	
	\hspace{1cm}$int$ $indice;$\\
	
	}

	$mochila\_frac1 (w,v,n,W)$\\

	\hspace{1cm}$for$ $i=0;$ $i<n;$ $i++$\\

	\hspace{2cm}$objetos[i] = cria\_objeto(v[i],w[i],i);$\\
	
	\hspace{2cm}$objetos\_selecionados[i] = 0;$\\

	\hspace{1cm}$mergesort\_densidade(objetos,n);$\\

	\hspace{1cm}$sum\_peso=0;$\\

	\hspace{1cm}$for$ $j=n-1;$ $j>=0;$ $j--$\\

	\hspace{2cm}$if$ $sum\_peso == W$\\
	
	\hspace{3cm}$return$ $objetos\_selecionados;$\\
	
	\hspace{2cm}$indice = objetos[j]->indice;$\\
	
	\hspace{2cm}$peso =  objetos[i]->peso;$\\

	\hspace{2cm}$if$ $peso + sum\_peso < W$\\
	
	\hspace{3cm}$objetos\_selecionados[indice] = peso;$\\
	
	\hspace{3cm}$sum\_peso+=peso;$\\
	
	\hspace{2cm}$else$\\
	
	\hspace{3cm}$objetos\_selecionados[indice] = W-sum\_peso;$\\
	
	\hspace{3cm}$sum\_peso+=W-sum\_peso;$\\
	
	\hspace{1cm}$return$ $objetos\_selecionados;$\\
	
	
	Com os elementos ornados por valor por peso, podemos facilmente sempre escolher quais resultarão em um maior valor na mochila.\\
	
	Inicializar e percorrer os vetores é realizado em $O(n)$. A ordenação com mergesort é realizada em $O(n$ $log n)$.\\
	
	\begin{center}
		O(c*n + n log n) = O(n log n)
\end{center}	 
	
	
	\subsection*{1.b)}
	
	\tab Encontra-se a mediana do valor por peso e particiona-se o vetor.\\
	
	\begin{itemize}
		\item Se a metade de maior valor cabe na mochila, todos os objetos são colocados na mochila e realiza-se o algoritmo na metade de menos valor. 
		
		\item Se a metade de maior valor não cabe na mochila, realiza-se o algoritmo na metade de maior valor.
		
		\item Se a metade de maior valor possui apenas um objeto, coloca-se a fração dele que cabe na mochila.
	\end{itemize}
	
	
	\textbf{Pseudo Código: }\\
	
	$struct$ $objeto${\\
	
	\hspace{1cm}$int$ $valor;$\\
	
	\hspace{1cm}$int$ $peso;$\\
	
	\hspace{1cm}$float$ $densidade;$\\
	
	\hspace{1cm}$int$ $indice;$\\
	
	}
	
	
	$mochila\_frac2 (w,v,n,W)$\\

	\hspace{1cm}$for$ $i=0;$ $i<n;$ $i++$\\

	\hspace{2cm}$objetos[i] = cria\_objeto(v[i],w[i],i);$\\
	
	\hspace{2cm}$objetos\_selecionados[i] = 0;$\\

	\hspace{1cm}$mochila\_frac2\_rec(objetos, W, 0, n, objetos\_selecionados);$\\

	\hspace{1cm}$return$ $objetos\_selecionados;$\\
	
	
	$mochila\_frac2\_rec(objetos,W,ini,fim, objetos\_selecionados)$\\

	\hspace{1cm}$if$ $ini > fim$\\

	\hspace{2cm}$return;$\\
	

	\hspace{1cm}$if$ $ini == fim$\\

	\hspace{2cm}$indice = objetos[ini]->indice;$\\
	
	\hspace{1cm}$if$ $objetos[ini]->peso >= W$\\
	
	\hspace{3cm}$objetos_selecionados[indice] = W;$\\
	
	\hspace{2cm}$else$\\
	
	\hspace{3cm}$objetos_selecionados[indice] =objetos[ini]->peso $\\
	
	\hspace{2cm}$return;$\\
	
	\hspace{1cm}$meio = ini+fim/2$\\
	
	\hspace{1cm}$k = kesima\_densidade(objetos, meio);$\\
	
	\hspace{1cm}$part\_inv\_densidade(ini,fim,objetos,k);$\\
	
	\hspace{1cm}$soma = somar\_peso(objetos,ini,meio);$\\
	
	\hspace{1cm}$if$ $soma > W$\\
	
	\hspace{2cm}$mochila\_frac2\_rec(objetos,W,ini, meio, objetos\_selecionados);$\\
	
	\hspace{1cm}$else$\\
	
	\hspace{2cm}$for$ $i=ini;$ $i<meio;$ $i++;$\\
	
	\hspace{3cm}$indice = objetos[i]->indice;$\\
	
	\hspace{3cm}$peso = objetos[i]->peso;$\\
	
	\hspace{3cm}$objetos\_selecionados[indice] = peso;$\\
	
	\hspace{2cm}$mochila\_frac2\_rec(objetos,W - soma,meio,fim, objetos\_selecionados)$\\
	
	
	Com o particionamento dos objetos pela mediana do valor por peso, asseguramos que sempre que uma metade é colocada na mochila, esta é a metade de maior valor.\\
	
	Achar a mediana, particionar e somar são realizados em $O(n)$.
Assim, temos a seguinte relação de recorrência:

 \[
    f(n)\leqslant\left\{
                \begin{array}{ll}
                     c,\: n = 1\\
                  f(\frac{n}{2}) + c'n, \: n>1
                \end{array}
              \right.
    \]            
  
           
   $Pelo\; Teorema\; Mestre:$
   
	\hspace{1cm}$a = 1$
	
	\hspace{1cm}$b = 2$
	
	\hspace{1cm}$k = 1$
	
	\hspace{1cm}$a < b^{k}$\\

	
	\hspace{5cm}$O(n^{k}) = O(n)$

   \subsection*{1.c)}
   
   \textbf{Pseudo Código: }\\
	
	$struct$ $objeto${\\
	
	\hspace{1cm}$int$ $valor;$\\
	
	\hspace{1cm}$int$ $peso;$\\
	
	\hspace{1cm}$float$ $densidade;$\\
	
	\hspace{1cm}$int$ $indice;$\\
	
	}
	
	$mochila\_frac3(w,v,n,W)$\\

	\hspace{1cm}$for$ $i=0;$ $i<n;$ $i++$\\

	\hspace{2cm}$objetos[i] = cria\_objeto(v[i],w[i],i);$\\
	
	\hspace{2cm}$objetos\_selecionados[i] = 0;$\\

	\hspace{1cm}$mochila\_frac3\_rec(objetos, W, 0, n, objetos\_selecionados);$\\

	\hspace{1cm}$return$ $objetos\_selecionados;$\\
	
	
	$mochila\_frac3\_rec(objetos,W,ini,fim, objetos\_selecionados)$\\

	\hspace{1cm}$if$ $ini > fim$\\

	\hspace{2cm}$return;$\\
	

		\hspace{1cm}$if$ $ini == fim$\\

	\hspace{2cm}$indice = objetos[ini]->indice;$\\
	
	\hspace{1cm}$if$ $objetos[ini]->peso >= W$\\
	
	\hspace{3cm}$objetos_selecionados[indice] = W;$\\
	
	\hspace{2cm}$else$\\
	
	\hspace{3cm}$objetos_selecionados[indice] =objetos[ini]->peso $\\
	
	\hspace{2cm}$return;$\\
	
	\hspace{1cm}$meio = ini+fim/2$\\
	
	\hspace{1cm}$valor_pivot = media_densidade(objetos, ini, fim);$\\
	
	\hspace{1cm}$part\_inv\_densidade(ini,fim,objetos,valor_pivot);$\\
	
	\hspace{1cm}$soma = somar\_peso(objetos,ini,meio);$\\
	
	\hspace{1cm}$if$ $soma > W$\\
	
	\hspace{2cm}$mochila\_frac3\_rec(objetos,W,ini, meio, objetos\_selecionados);$\\
	
	\hspace{1cm}$else$\\
	
	\hspace{2cm}$for$ $i=ini;$ $i<meio;$ $i++;$\\
	
	\hspace{3cm}$indice = objetos[i]->indice;$\\
	
	\hspace{3cm}$peso = objetos[i]->peso;$\\
	
	\hspace{3cm}$objetos\_selecionados[indice] = peso;$\\
	
	\hspace{2cm}$mochila\_frac3\_rec(objetos,W - soma,meio,fim, objetos\_selecionados)$\\
	
	
	Agora podemos ter um caso de particionamento desbalanceado, onde sempre um lado tem 1 elemento e o outro tem n-1 elementos. Se todos elementos menos uma fração do último cabem na mochila, todos serão percorridos. Assim, teremos intâncias de $n-1, n-2, ..., 1$, elementos. Portanto a soma de todas as iterações é igual a $\frac{n-1(n-2)}{2}.$
	
	\begin{center}
		$O(\frac{n-1(n-2)}{2}) = O(n^{2})$
	\end{center}
	
	
	
	\subsection*{Resultados}

	\vspace{0.5cm}	
	
	\begin{center}
		\begin{tabular}{|c|c|c|c|c|}
			\hline
			Nome do Arquivo & Num exec & cpu/Alg 1 & cpu/Alg 2 & cpu/Alg 3 \\
			\hline
			m50.in & 10000 &  0.000309375  &  0.00073125 & 0.00008125\\
			\hline 
			m58.in & 8621 &    0.0003842362    &    0.0008808433    & 0.00007793469\\
			\hline 
			m67.in & 7463 &    0.0004438564    &    0.0009526162    &    0.0001046831\\
			\hline 
			m78.in & 6411 &    0.0003436476    &    0.001013882     &    0.000112112\\
			\hline 
			m91.in &  5495 &    0.0006596906    &    0.001632166     &    0.0001393312\\
			\hline 
			m106.in &  4717 &    0.0007751219    &    0.00190468      &    0.0001192495\\
			\hline 
			m124.in &  4033  &    0.0009182061    &    0.002405932     &    0.0001820915\\
			\hline 
			m145.in & 3449 &    0.0006886054    &    0.001857422     &    0.0001948028\\
			\hline 
			m169.in & 2959  &    0.001378211     &    0.003284471     &    0.000264025\\
			\hline 
			m197.in & 2539  &    0.001643117     &    0.003877019     &    0.0002338519\\
			\hline 
			m230.in &  2174  &    0.001918986     &    0.004585442     &    0.0003449862\\
			\hline 
			m269.in &  1859  &    0.001395239     &    0.003353618     &    0.0003446073\\
			\hline 
			m314.in &  1593 &    0.002854284     &    0.006287272     &    0.0004413842\\
			\hline 
			m367.in & 1363  &    0.003335932     &    0.007348221     &    0.0004126926\\
			\hline 
			m429.in &  1166  &    0.003953152     &    0.008871141     &    0.0006164237\\
			\hline 
			m501.in &  999 &    0.002846597     &    0.006475225     &    0.000609985\\
			\hline 
			m586.in &  854  &    0.005818208     &    0.01266101      &    0.0008233314\\
			\hline 
			m685.in &  730 &    0.006913527     &    0.01502568      &    0.0007491438\\
			\hline 
			m801.in & 625 &    0.0081          &    0.01705         &    0.00115\\
			\hline 
			m937.in &  534  &    0.005559457     &    0.01331344      &    0.001228933\\
			\hline 
			m1096.in & 457  &    0.01182987      &    0.02424097      &    0.001572757\\
			\hline 
			m1282.in &  391 &    0.01414642      &    0.02709399      &    0.001478581\\
			\hline 
			m1499.in &  334 &    0.01665419      &    0.03424401      &    0.002198728\\
			\hline 
			m1753.in &  286 &    0.0111451       &    0.02545892      &    0.002239948\\
			\hline 
			m2051.in &  244  &    0.02407787      &    0.04482582      &    0.003009734\\
			\hline 
			m2399.in &  209  &    0.02900718      &    0.05554725      &    0.002766148\\
			\hline 
			m2806.in &  179  &    0.03421788      &    0.06407123      &    0.004189944\\
			\hline 
			m3283.in &  153  &    0.0223652       &    0.04381127      &    0.004289216\\
			\hline 
			m3841.in &  131 &    0.0492605       &    0.08277672      &    0.005605916\\
			\hline 
			m4493.in &  112  &    0.05929129      &    0.1060268       &    0.005580357\\
			\hline 
			m5256.in &  96 &    0.0703125       &    0.1220703       &    0.008463542\\
			\hline 
			m6149.in &  82  &    0.04668445      &    0.0929878       &    0.008384146\\
			\hline 
			m7194.in & 70  &    0.1008929       &    0.1747768       &    0.01183036\\
			\hline 
			m8416.in & 60  &    0.1208333       &    0.2036458       &    0.01041667\\
			\hline 
			m9846.in & 51  &    0.1433824       &    0.2352941       &    0.01593137\\
			\hline 
			m11519.in &  44  &    0.09019886      &    0.1420455       &    0.01669034\\
			\hline 
			m13477.in &  38  &    0.2088816       &    0.3322368       &    0.02179276\\
			\hline 
			m15768.in & 32  &    0.2524414       &    0.3867188       &    0.02148438\\
			\hline 
			m18448.in & 28  &    0.3024554       &    0.4665179       &    0.03459821\\
			\hline 
			m21584.in &  24  &    0.1855469       &    0.4733073       &    0.03515625\\
			\hline 
			m25253.in &  20 &    0.440625        &    0.65            &    0.05\\
			\hline 
			m29546.in &  17 &    0.5303309       &    0.7702206       & 0.05238971\\
			\hline 
			m34568.in & 15 &    0.634375        &    0.8895833       & 0.078125\\
			\hline 
			m40444.in &  13 &    0.3858173       &    0.8389423       &    0.07451923\\
			\hline 
			m47319.in & 11  &    0.9488636       &    1.227273        &    0.1122159\\
			\hline 
			m55363.in & 10  &    1.157812        &    1.45625         &    0.1015625\\
			\hline 
			m64774.in & 8  &    1.386719        &    1.730469        &    0.1640625\\
			\hline 
			m75785.in & 7  &    0.8415179       &    1.366071        &    0.1495536\\
			\hline 
			m88668.in & 6  &    2.125           &    2.416667        &    0.2265625\\
			\hline 
			m103741.in & 5  &    2.559375        &    2.828125        &    0.078125\\
			\hline 

		
		\end{tabular}
	\end{center}
	
	É possível observar que o terceiro algoritmo é sempre o de menor tempo. Acreditamos que os possíveis motivos para que isso ocorra são:
	
	\begin{itemize}
		\item A função do kesimo e/ou o MergeSort foram implementados de forma ineficiente.
		\item O pior caso descrito na última questão não ocorre.
		\item Embora o algoritmo que calcula o kesimo seja linear, a constante pela qual n é multiplicado é muito grande.
	\end{itemize}


\end{document}