\documentclass[10pt,a4paper]{article}
\usepackage[utf8]{inputenc}
\usepackage[portuguese]{babel}
\usepackage[T1]{fontenc}
\usepackage{amsmath}
\usepackage{amsfonts}
\usepackage{amssymb}
\usepackage{geometry}
\usepackage{tabto}


\begin{document}


\begin{flushleft}

	\large{Análise de Algoritmos - Trabalho 1}\\
	Clara de Mattos Szwarcman - 1310351\\
	Lucas Ribeiro Borges - \\
	Guilherme Simas Abinader -\\ 

\end{flushleft}


	\section*{1 - Controle de Qualidade na Produção de Frascos de Vidro}
	
		\vspace{1cm}
	
		\subsection*{1)}
		
		\tab A altura será dividida em raiz de n intervalos, aonde cada intervalo possui raiz de n degraus. O primeiro frasco será jogado de raiz de n em raiz de n degraus. Quando o frasco quebrar, jogaremos o segundo frasco a partir do início desse intervalo de degrau em degrau até que ele quebre.\\
		
		\textbf{Pseudo Código: }\\
		
		$Degrau\_2\_frascos(x,n)$\\

		\hspace{1cm} $raiz\_n = sqrt(n);$\\


		\hspace{1cm} $for$  $i = 0;$ $i < n;$ $i+=raiz\_n$\\

		\hspace{2cm} $if$ $i >= x$\\

		\hspace{3cm} $for$ $j = i-raiz_n;$ $j < i;$ $j++$\\

		\hspace{4cm} $if$ $j == x$\\

		\hspace{5cm} $return$ $j;$\\


\vspace{1cm}

	Quando o primeiro frasco quebrar teremos um intervalo de tamanho $ \sqrt{n} $ que com certeza contém a altura em que o frasco quebra, visto que se o frasco não quebrou no degrau anterior ao início do intervalo, ele também não quebra em nenhum dos degraus abaixo dele. Assim, ao percorrermos o intervalo de um em um, encontraremos a altura x.\\
	
	No pior dos casos, o frasco quebra no último degrau, portanto, o primeiro frasco será jogado de todos os intervalos $(\sqrt{n}$ vezes.$)$ Como ele quebrou no último degrau, será conferido se ele quebra em algum degrau pertencente ao último intervalo. Dessa maneira, o segundo frasco será jogado em cada degrau do intervalo $(\sqrt{n}$ vezes$)$, até finalmente quebrar no último degrau. Sendo assim, foi jogado $2*\sqrt{n}$.\\
	
	\begin{center}
		$O(2*\sqrt{n}) = O(\sqrt{n})$	
	
	\end{center}
	
	
	\subsection*{2)}
	
	\vspace{0.5cm}
	
	\tab Tendo 3 frascos:\\

\hspace{1cm} O primeiro frasco será jogado em intervalos de $n^{\frac{2}{3}}$ até quebrar. O segundo frasco será jogado em intervalos de $n^\frac{1}{3}$, no intervalo de $n^{\frac{2}{3}}$ encontrado. O terceiro frasco será jogado de degrau em degrau no intervalo de $n^{\frac{1}{3}}$ encontrado.\\


Tendo 4 frascos:\\

\hspace{1cm} O primeiro frasco será jogado em intervalos de $n^{\frac{3}{4}}$ até quebrar. O segundo frasco será jogado em intervalos de $n^{\frac{2}{4}}$, no intervalo de $n^{\frac{3}{4}}$ encontrado. O terceiro frasco será jogado em intervalos de $n^{\frac{1}{4}}$, no intervalo de $n^{\frac{2}{4}}$ encontrado. O quarto frasco será jogado de degrau em degrau no intervalo de $n^{\frac{1}{4}}$ encontrado.\\

Tendo k frascos:\\

 \hspace{1cm} A altura $((\sqrt[k]{n})^{k}$ degraus$)$é dividida em $\sqrt[k]{n}$ intervalos de tamanho $(\sqrt[k]{n})^{k-1}$. Jogamos o primeiro frasco em intervalos de $(\sqrt[k]{n})^{k-1}$ degraus. Quando o frasco quebrar, o último intervalo $((\sqrt[k]{n})^{k-1}$ degraus$)$ será dividido em $\sqrt[k]{n}$ intervalos de tamanho $(\sqrt[k]{n})^{k-2}$ degraus e o segundo frasco será jogado em intervalos de $(\sqrt[k]{n})^{k-2}$ degraus. Isso ocorrerá sucessivamente para todos os k frascos. No frasco k teremos um intervalo de tamanho $(\sqrt[k]{n})^{k-(k-1)}$, que é igual a $\sqrt[k]{n}$. O frasco será jogado em intervalos de $(\sqrt[k]{n})^{k-k}$, ou seja, de degrau em degrau, até quebrar.\\ 

 
\textbf{Pseudo Código: }\\


	$Degrau\_k\_frascos(x,n,k)$\\
	
	\hspace{1cm}$raiz\_kesima = raiz(n,k)$\\
	
	\hspace{1cm}$inicio = 0;$\\
	
	\hspace{1cm}$fim = n;$\\
	
	\hspace{1cm}$incremento = pow(raiz\_kesima,k-1)$\\
	
	\hspace{1cm}$for$ $i=0;$ $i < k;$ $i++$\\
	
	\hspace{2cm}$for$ $j = inicio;$ $j < fim;$ $j+= incremento$\\
	
	\hspace{3cm}$if$ $j >= x$\\
	
	\hspace{4cm}$if$ $incremento == 1$\\
	
	\hspace{5cm}$return$ $j;$\\
	
	\hspace{4cm}$inicio = j - incremento ;$\\
	
	\hspace{4cm}$fim = j;$\\
	
	\hspace{4cm}$incremento = incremento/raiz\_kesima;$\\
	
	\hspace{4cm}$break;$\\
	
	
	Segue a premissa do primeiro, aonde sempre se tem certeza do intervalo em que ocorre a quebra, porém com mais frascos para serem utilizados. Portanto, podemos inicialmente dividir a altura em intervalos maiores com mais subdivisões, assim postergando a procura de um em um, que será feita em um intervalo menor.\\
	
	Para cada frasco estamos realizando no máximo $\sqrt[k]{n}$ testes, visto que para o frasco i temos um espaço de $(\sqrt[k]{n})^{k-(i-1)}$ degraus e o jogaremos em intervalos de $(\sqrt[k]{n})^{k-i}$ degraus. Como temos k frascos, isso será realizado k vezes. Assim, o número total de quedas será $k * \sqrt[k]{n}$.\\
	
	\begin{center}
	
		$O(k*\sqrt[k]{n})$\\
	\end{center}
	
	Se k, é um número fixo, a complexidade será O($\sqrt[k]{n}$), como provado anteriormente.\\
	
	
	\subsection*{3)}
	
	\vspace{0.5cm}
	
	\tab A menor complexidade assintótica possível é de $O(log n)$.\\

O algoritmo realiza uma busca binária ao longo da escada, jogando um frasco a cada comparação, se o frasco quebra, busca-se na metade inferior, do contrário busca-se na metade superior. Quando o intervalo é de 1 degrau, podemos garantir que encontramos a altura correta.\\


\textbf{Pseudo Código: }\\

$Degrau\_logn\_frascos(x,n)$\\
					

	\hspace{1cm}$busca\_binaria(x,n)$

	


\end{document}